\documentclass[a4paper]{article}
\usepackage[left=2cm, right=2cm, top=0.8in, bottom=1.0in]{geometry}
\usepackage{fontspec}
\usepackage{xeCJK}
\usepackage{indentfirst}
\usepackage{setspace}
\usepackage{paralist}
\usepackage{fancyhdr}
\usepackage[ruled, linesnumbered]{algorithm2e}
\usepackage{setspace}
\usepackage[backref]{hyperref}
\usepackage{amsthm,amsmath,amsfonts,amssymb}
\usepackage{graphicx}
\usepackage{listings,xcolor}
\usepackage{inconsolata}
\usepackage{tikz,forest}
\usepackage{caption, subcaption, amsfonts, dcolumn}
\usepackage{booktabs, multirow, bigstrut, makecell}
\usepackage{lscape}
\definecolor{mygreen}{rgb}{0,0.6,0}  
\definecolor{mygray}{rgb}{0.5,0.5,0.5}  
\definecolor{mymauve}{rgb}{0.58,0,0.82}  
\pagestyle{fancy}
\fancyhf{}
\fancyhead[L]{HTTP-SERVER实验报告 \quad 郭宇祺 \quad 卫思为} %页眉
\fancyfoot[C]{\thepage}
\let\itemize\compactitem
\let\enditemize\endcompactitem
\let\enumerate\compactenum
\let\endenumerate\endcompactenum
\let\description\compactdesc
\let\enddescription\endcompactdesc
\setstretch{1.35} 
\setlength{\parindent}{2em} 
\setmainfont{Times New Roman}
\setCJKmainfont[BoldFont={黑体}, ItalicFont={楷体}]{宋体}
\setCJKsansfont{黑体}
\setCJKmonofont{仿宋}
\renewcommand{\arraystretch}{1.3}
\renewcommand{\figurename}{图}
\renewcommand{\thesubfigure}{(\alph{subfigure})}


\lstset{ %  
  backgroundcolor=\color{white},   % choose the background color; you must add \usepackage{color} or \usepackage{xcolor}  
  basicstyle=\footnotesize\setstretch{1}\setmainfont{Courier New}\bfseries ,        % the size of the fonts that are used for the code  
  breakatwhitespace=true,         % sets if automatic breaks should only happen at whitespace  
  breaklines=true,                 % sets automatic line breaking  
  captionpos=bl,                    % sets the caption-position to bottom  
  commentstyle=\color{mygreen},    % comment style  
  escapeinside={\%*}{*)},          % if you want to add LaTeX within your code  
  extendedchars=true,              % lets you use non-ASCII characters; for 8-bits encodings only, does not work with UTF-8  
  frame=single,                    % adds a frame around the code  
  keepspaces=true,                 % keeps spaces in text, useful for keeping indentation of code (possibly needs columns=flexible)  
  keywordstyle=\color{blue},       % keyword style  
  numbers=left,                    % where to put the line-numbers; possible values are (none, left, right)  
  numbersep=5pt,                   % how far the line-numbers are from the code  
  numberstyle=\tiny\color{mygray}, % the style that is used for the line-numbers  
  rulecolor=\color{black},         % if not set, the frame-color may be changed on line-breaks within not-black text (e.g. comments (green here))  
  showspaces=false,                % show spaces everywhere adding particular underscores; it overrides 'showstringspaces'  
  showstringspaces=false,          % underline spaces within strings only  
  showtabs=false,                  % show tabs within strings adding particular underscores  
  stepnumber=1,                    % the step between two line-numbers. If it's 1, each line will be numbered  
  stringstyle=\color{orange},     % string literal style  
  tabsize=2,                       % sets default tabsize to 2 spaces  
}  
\newtheorem{theorem}{定理}
\begin{document} 
\title{HTTP-SERVER实验报告}
\author{小组成员:郭宇祺 \; 卫思为}
\date{}
\maketitle
\normalsize
\section{实验内容简介}
我们的HTTP服务器实现了三个基本功能:服务器主页获取、文件上传、文件下载。我们在服务器主页获取和文件下载这两个功能上实现了HTTP GET方法,在文件上传这个功能上实现了HTTP POST方法以及分块传输功能。同时我们针对文件上传功能实现了HTTP pipeline。我们还借助openssl库实现了SSL加密,目前我们的服务器只能通过https协议访问,不能通过普通http协议访问。
\section{实现方法}
\subsection{HTTP POST/GET方法}
在本次实验中,由于我们的http-server使用纯C语言编写,因此无法使用高级框架解析请求头,也很难使用高级数据结构存储解析得到的信息。因此,我们使用较为简单的字符串匹配的方法手动实现HTTP请求头的解析。在我们的实现中,我们所实现的HTTP请求头解析功能只会解析与本次实验内容相关的项,对于其他无关项,我们的请求头解析功能不做识别和处理。

HTTP GET方法的实现较为简单。在HTTP请求头中,第一行就包含了本次HTTP请求所访问的URI信息。在http协议中,对于URI,使用符号"?"作为分隔符分隔访问路径与参数,"?"之前的字符串表示所要访问的文件/应用所在路径,之后的字符串表示访问对应文件/应用时所需要的参数。参数以键值对key=value的形式出现,多个参数之间使用"\&"符号进行分隔。基于以上原则,我们编写代码实现了HTTP GET方法的解析功能,相关代码主要集中在lib/server\_handler.c文件中的cut\_params()函数和get\_value()函数中。需要注意的是,由于URI可能包含非英文字符,在使用之前需要先进行解码,解码代码位于lib/safe\_connect.c文件中的urldecode()函数中。

HTTP POST方法的实现略微复杂。与HTTP GET方法不同,HTTP POST方法将请求参数放置于请求头的请求项中。对于文件上传功能,请求头会在Content-Length项中标识后续请求长度,并在boundary项中标识包裹文件内容所使用的特殊字符串。在每一个被boundary所指示的字符串所包裹的块中,都存在一个filename项,其指定了当前块内容所对应的文件名。根据以上信息,我们编写了HTTP POST解析功能,相关代码主要集中在lib/server\_handler.c文件中的get\_value()函数中。

\subsection{文件上传、下载和分块传输}
在服务器主页中,我们创建了如下的表单,以供用户选择文件并上传:
\begin{lstlisting}[language=html]
  <form action="/upload" method="post" enctype="multipart/form-data">
  上传文件: <input type="file" name="upload_filename"><br>
  <input type="submit">
  </form>
\end{lstlisting}
当用户点击上传按钮后,浏览器会创建POST请求向服务器传送文件内容。我们的服务器在完成请求头解析、文件内容接受等一系列内容后,将接收到的文件保存到resources文件夹中,同时向浏览器返回重定向信息,将页面重新导向服务器主页。并在页面最上端显示“xx文件上传成功”的提示信息。文件上传的处理函数位于lib/server\_handler.c文件中的file\_upload()函数中。

同时,我们还在服务器主页中,添加了服务器上已有文件的信息,并将每一条信息做成如下所示的超链接:
\begin{lstlisting}[language=html]
  <a href=\"/download?filename=${FILENAME}\">${FILENAME}</a><br>
\end{lstlisting}

每当用户点击对应的文件名,浏览器就会自动向服务器发送请求,开始文件的下载功能。我们的服务器在返回的请求头中设置如下两项,以便触发浏览器的下载功能:
\begin{lstlisting}[language=C]
  Content-Type: application/octet-stream
  Content-Disposition: attachment;filename=${FILENAME}
\end{lstlisting}

文件下载的处理函数位于lib/server\_handler.c文件中的file\_download\_chunked()函数中。

% TODO
TODO: 分块传输


\subsection{HTTP持久连接和管道}

\subsection{使用openssl库实现HTTPS}

\subsection{使用libevent实现多路并发}
我们首先创建一个事件listen\_fd,指定其监视服务器套接字server\_fd,并设置其在接收到新连接时的回调函数为on\_accept(),相关代码如下:
\begin{lstlisting}[language=C,title=server.c]
  struct event listen_ev;
	base = event_base_new();
	event_set(&listen_ev, server_fd, EV_READ | EV_PERSIST, on_accept, NULL);
	event_base_set(base, &listen_ev);
	event_add(&listen_ev, NULL);
	event_base_dispatch(base);
\end{lstlisting}

随后,在函数on\_accept()中,我们为服务器建立与客户端之间的连接,并处理相关请求,相关代码如下:
\begin{lstlisting}[language=C, title=server.c]
  void on_accept(int server_fd, short event, void *arg) 
{
	struct sockaddr_in client_addr;
	socklen_t client_addr_size = sizeof(client_addr);
	int client_fd;
	char recv_buffer[DEFAULT_RECV_BUFFER_SIZE];
	int n;
	char reqs[N_REQ][DEFAULT_RECV_BUFFER_SIZE] = {0};
	int recv_rest = 0;
	// read_ev must allocate from heap memory, otherwise the program would crash from segmant fault
	if ((client_fd = accept(server_fd, (struct sockaddr *)&client_addr,
							&client_addr_size)) == -1)
	{
		perror("accept failed:");
    return;
	}

	SSL *ssl = SSL_new(ctx);
	SSL_set_fd(ssl, client_fd);

	if (SSL_accept(ssl) <= 0)
	{
		perror("ssl state:");
	}
	memset(recv_buffer, 0, sizeof(char) * DEFAULT_RECV_BUFFER_SIZE);
  while (1)
	{
		if (n == 0)
			n = recv_s(ssl, recv_buffer + recv_rest, DEFAULT_RECV_BUFFER_SIZE - recv_rest, 0);
		if (n == 0)
			break;

		int n_buffer;
		int req_len[N_REQ] = {0};

		memset(reqs, 0, sizeof(char) * N_REQ * DEFAULT_RECV_BUFFER_SIZE);
		n_buffer = divide_buffer(recv_buffer, n, reqs, req_len, &recv_rest);

		for (int i = 0; i < n_buffer; i++)
		{
			handle(ssl, reqs[i], req_len[i]);
		}
		memmove(recv_buffer, recv_buffer + n - recv_rest, recv_rest);
		n = recv_rest;
	}
	SSL_shutdown(ssl);
	SSL_free(ssl);

	close(client_fd);
}

\end{lstlisting}

\section{功能测试}
\subsection{文件上传下载测试}
我们已经在resources文件夹中放置了一些文件,包含各种格式类型(pdf、jpg、pptx、docx、xlsx等)。我们将这些文件上传后下载,并重新打开文件验证,发现这些文件的内容和格式都没有损坏。与此同时,服务器的主页也能够正确显示当前服务器上所存储的文件信息,如图\ref{fig:homepage}所示:
\begin{figure}[!h]
  \centering
  \includegraphics[width=0.4\textwidth]{homepage.jpg}
  \caption{服务器主页示意图}
  \label{fig:homepage}
\end{figure}

因此服务器的文件上传下载功能运行良好。
\subsection{持久连接和管道测试}

\subsection{HTTPS测试}

\subsection{多路并发测试}

\end{document}